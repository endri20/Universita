%%There is just one section and two subsections.
\section{Esercizio 1}
 
\subsection{Esplorate il vostro file system. \\
Qual \`e il pathname della vostra home directory?}

\textbf{Da terminale:}	\\
\verb:> cd:	 \\
\verb:> pwd:  \\
\verb:/Users/Aldo:  \\


\subsection{Visualizzate i file della vostra home directory ordinati in base alla data di ultima modifica.}

\textbf{Da terminale:}		\\
\verb:> ls -t:	\\
 

\subsection{Che differenza c’\`e tra i comandi cat, more, tail?}

- \textbf{cat}: concatena i file, stampando il risultato
dell'operazione.	\\
- \textbf{more}: filtro che serve a visualizzare sullo schermo un
flusso di testo, una pagina per volta.	\\
- \textbf{tail}: stampa su standard output le ultime 10 righe dei
file, forniti come argomenti.	\\


\subsection{Trovate un modo per ottenere l’elenco delle subdirectory contenute ricorsivamente nella vostra home.}

\textbf{Da terminale:}		\\
\verb:> ls -R:	\\


\subsection{I seguenti comandi che effetto producono? Perch\`e?}
\texttt{> cd	\\
> mkdir d1	\\
> chmod 444	d1	\\
> cd d1	\\}
\\
- \textbf{cd}: la directory corrente diventa quella della home	\\
- \textbf{mkdir d1}: crea la directory di nome "dl" nella directory
corrente	\\
- \textbf{chmode 444 d1}: assegna permessi \verb:r--r--r--: a d1	\\
- \textbf{cd d1}: la directory corrente dovrebbe diventare d1 (se
si hanno i permessi)	\\