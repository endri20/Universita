%%There is just one section and two subsections.
\section{Esercizio 2}

\subsection{Scrivete un unico comando (pipeline) per: \\
1) copiare il contenuto della directory dir1 nella directory dir2; \\
2) fornire il numero di file (e directory) a cui avete accesso, contenuti
ricorsivamente nella directory studenti (si pu\`o utilizzare ''ls -R'' e il
comando ''find''); \\
3) fornire la lista dei file della home directory il cui nome \`e una stringa di
 3 caratteri seguita da un numero.}
1) \verb: > cp ~r dir1 dir2: \\
2) \verb: > find ~ 2>/dev/null | wc -l: \\
3) \verb: > ls ~/???[0-9]: \\


\subsection{Qual \`e la differenza tra i seguenti comandi?}
\verb:> ls:	\\
\verb:> ls | cat:	\\
\verb:> ls | more:	\\
\\
- \verb:ls:: stampa la lista dei file contenuti nella directory corrente,
disposti in righe e colonne in ordine alfabetico. \\
- \verb:ls | cat:: stampa la lista dei file contenuti nella directory
corrente disposti su un unica colonna in ordine alfabetico. \\
- \verb:ls | more:: stampa la lista dei file contenuti nella directory
corrente disposti in ordine alfabetico, in colonna e organizzando l'output su
schermo in pagine. \\


\subsection{Quale effetto producono i seguenti comandi?}
\verb:> uniq < file (dove file `e il nome di un file): \\
\verb:> who | wc -l:	\\
\verb:> ps -e | wc -l:	\\
\\
- \verb:uniq < file:: stampa il contenuto del file, sostituentdo le
linee adiacenti uguali con un'unica occorrenza. \\
- \verb:who | wc -l:: Stampa il numero di utenti collegati al
sistema. \\
- \verb:ps -e | wc -l:: Stampa il numero di processi + 1; il +1 è la
linea intestazione prodotta da ls. \\


\subsection{Ridefinire il comando ''rm'' in modo tale che non sia chiesta
conferma prima della cancellazione dei file}
\verb:> alias rm='rm -f': \\


\subsection{Definire il comando ''rmi'' (rm interattivo) che chiede conferma
prima di rimuovere un file.}
\verb:> alias rmi='rm -i': \\


\subsection{Sapendo che il comando ''ps'' serve ad elencare i processi del
sistema, scrivere una pipeline che fornisca in output il numero di tutti i processi in esecuzione.}
\verb:> ps -e --no-headers | wc -l: \\


\subsection{Salvare in un file di testo l’output dell’ultimo evento contenente
il comando ''ls''.}
\verb:> ls | more > Scrivania/relazioni/file_out_ls.txt: \\


\subsection{Scrivere un comando che fornisce il numero dei comandi contenuti nella history list}
\verb:> history | wc -l: \\


\subsection{Scrivere un comando che fornisce i primi 15 comandi della history list}
\verb:> history | head -15: \\


\subsection{Quali sono i comandi Unix disponibili nel sistema che iniziano con
''lo''?}
\verb:> lo + il tasto 'tab' 2 volte: \\


\subsection{Fornire almeno due modi diversi per ottenere la lista dei file della
vostra home directory il cui nome inizia con ''al''.}
\verb:> ls ~/al il tasto 'tab' + il tasto 'tab': \\
\verb:> ls al*: \\


\subsection{Qual \`e l’effetto dei seguenti comandi?}
\verb:> ls -R || (echo file non accessibili > tmp): \\
\verb:> (who | grep rossi) && cd ~rossi: \\
\verb:> (cd / ; pwd ; ls | wc -l ):	\\
\\
\verb:- ls -R || (echo file non accessibili > tmp):: stampa la lista ricorsiva
dei file contenuti nella directory corrente; nel caso in cui ls -R
fallisca, stampa il messaggio 'file non accessibili' nel file tmp nella
directory corrente. \\
\verb:- (who | grep rossi) && cd ~rossi:: la directory corrente cambia in quella
dell'utente rossi (se questo è collegato al sistema)
\\
\verb:- (cd / ; pwd ; ls | wc -l ):: cambia la directory corrente in /, la
stampa (con 'pwd') e stampa il numero di elementi contenuti nella dir corrente \\
