%%There is just one section and two subsections.
\section{Esercizio 4}

\subsection{Progettare uno script drawsquare che prende in input un parametro intero
con valore da 2 a 15 e disegna sullo standard output un quadrato.}

Esempio:
\begin{lstlisting}
> drawsquare 4
+ - - +
|     |
|     |
+ - - +
\end{lstlisting}

\textbf{\\Codice:}
\begin{lstlisting}
if test $# -ne 1
   then
   echo 'Utilizzo dello script: drawsquare <n>'
   exit 1
fi

if test $1 -le 2 -o $1 -ge 51
   then
   echo 'Il parametro deve essere un numero in [3;50]'
   exit 2
fi
x=$1 
y=$1 
while test $y -gt 0
   do
   while test $x -gt 0
   	 do
	 if test $x -eq 1 -o $x -eq $1
	    then 
	    if test $y -eq 1 -o $y -eq $1
	       then
	       echo -n "+ "
	    else
	       echo -n "| "
	    fi
	 else 
	      if test $y -eq 1 -o $y -eq $1
		 then
		 echo -n "- "
	      else
		 echo -n "  "
	      fi
	 fi
	 x=$[$x-1]
   done
   x=$1
   y=$[$y-1]
   echo
done
exit 0
\end{lstlisting}


\subsection{Progettare uno script che prende in input come parametro il nome di una directory e cancella tutti i file con nome core dall’albero di directory con radice la directory parametro.}

\textbf{\\ Codice:}
\begin{lstlisting}
if test $# -ne 1
then
	echo 'Utilizzo dello script: rmcore <dir>'
	exit 1
fi

if ! test -d $1 # Gestione degli errori.
then
	echo "$1 deve essere una directory"
	exit 2
fi

find $1 -name core -exec rm {} \; 2>/dev/null

exit 0
\end{lstlisting}


\subsection{Progettare uno utility-script processi per la gestione user-friendly
dei processi in memoria. Le operazioni di base sono gestite tramite un semplice
menu testuale con input da tastiera. \\
- Visualizzazione processi di un utente selezionato (PID, CPU e riga comando che
lo ha generato) \\
- Implementare una versione di top minimale, con un ordinamento in base all’uso
CPU, e con la fotografia del momento, senza aggiornamenti. \\
- Possibilit\`a di kill -9 su un processo utente, su tutti i processi (sempre
con un comodo menu) \\
- Ogni funzione attivata ritorna nel menu principale di
scelta}

\textbf{\\ Codice:}
\begin{lstlisting}
echo "MENU: ";
echo "1) Visualizza i processi di un utente selezionato";
echo "2) Top minimale (con ordinamento in base alla CPU)";
echo "3) Kill -9 su un processo utente, su tutti i processi";
echo " ";
echo "Scegli una delle opzioni: ";

read opt;
if [ $opt -lt 1 -o $opt -gt 4 ]
then
	echo "Scelta non valida!";
fi

case $opt in
	1) echo "inserisci il nome di un utente:"
		read username; ps -u $username ;;
	2) top ;;
	3) echo "inserisci il PID del processo da eliminare:";
		read pid;kill -9 $pid ;;
esac
\end{lstlisting}